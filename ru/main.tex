\input head

\begin{document}

\section{Содержание архива}
\begin{enumerate}
    \item Генерация тестов, сюда может входить Generator, Validator, Jury solution
        в зависимости от конкретного формата задачи
    \item Interactor (может использоваться системный,
        в этом случае необходимо явно указать идентификатор системного Interactor'а)
    \item Checker (может использоваться системный, а также может отсутствовать, если его роль выполняет Interactor)
\end{enumerate}

\pagebreak

\section{Проверка решений}
\begin{enumerate}
    \item Система запускает Interactor и решение и перенаправляет их стандартные потоки друг на друга
    \item Interactor, используя данные теста, организует взаимодействие с решением, генерируя \textit{Interactor's output} при необходимости
    \item При наличии, Checker выносит вердикт, используя данные теста и \textit{Interactor's output}
\end{enumerate}

% TODO Добавить информацию про Tokenizer

\vspace{2cm}

\includepicture{execution}{Проверка}{1}

\pagebreak

\section{Interactor}
Interactor обменивается данными с решением используя STDIN/STDOUT.
Пути до файлов с тестами, а также \textit{Interactor's output} передаются
Interactor'у через окружение (см. man 3 getenv).
Файлы тестов не являются обязательными, в этом случае
соответствующие им переменные окружения установлены не будут.

\begin{itemize}
    \item Файлы теста -- не являются обязательными
        \begin{itemize}
            % TODO JUDGE\_TEST\_ prefix + auto data\_id translation
            \item JUDGE\_TEST\_IN -- входной файл теста
            \item JUDGE\_TEST\_HINT -- файл-подсказка, может содержать к примеру правильные ответы
        \end{itemize}
    \item JUDGE\_INTERACTOR\_OUTPUT -- \textit{Interactor's output}
    \item JUDGE\_INTERACTOR\_CUSTOM\_MESSAGE -- дополнительное сообщение для вердикта \textit{Custom failure}
\end{itemize}

\section{Возможные вердикты}
\begin{itemize}
    \item OK -- Interactor завершился успешно с exit code = 0
    \item FAILED -- Interactor не завершился (exit code не установлен, завершён сигналом)
    \item TIME\_LIMIT\_EXCEEDED -- решение участника потребляет слишком много CPU time
    \item REAL\_TIME\_LIMIT\_EXCEEDED -- решение участника работает слишком долго (здесь учитывается также время простоя)
    \item TODO Real time limit для отдельных взаимодействий (требует Tokenizer)
    \item MEMORY\_LIMIT\_EXCEEDED -- решение участника расходует слишком много памяти
    \item ABNORMAL\_EXIT -- решение участника завершилось с ненулевым exit code или было завершено сигналом

    \item PRESENTATION\_ERROR -- решение участника вывело запрос в неправильном формате

        Возвращает Tokenizer или Interactor.

        Пример: Interactor ждет два числа, а ему пришла строка.

    \item WRONG\_ANSWER -- решение участника нашло неправильный ответ
        Возвращает Interactor или Checker.

    \item QUERIES\_LIMIT\_EXCEEDED -- решение участника выводит слишком много запросов.

        Возвращает Interactor, Tokenizer.
        Системное ограничение.

    \item CUSTOM\_FAILURE

        Interactor сигнализирует о специфической для задачи ошибке.
        Комментарий доступен пользователю.

    \item INCORRECT\_REQUEST -- решение вывело данные, соответствующие протоколу, но некорректные по условию.
        Выдаётся Interactor'ом.

    \item INSUFFICIENT\_DATA -- решение вывело недостаточно данных.

        Например, завершилось, не выдав ответ.
        Возвращает Interactor или Tokenizer.
\end{itemize}

\section{Тестирование задач}
Участнику должна быть предоставлена возможность тестировать свои решение на сервере.
Для этого к задаче должны прилагаться несколько тестов жюри (около 10).
Эти тесты необязательно должны соблюдать формат официальных тестов
(например, в задаче $n = 10^5$, а участнику нужно проверить решение при маленьких n).

После запуска на сервере ему должна быть предоставлена следующая информация:
\begin{itemize}
    \item вердикт
    \item время
    \item память
    \item вывод решения
    \item вывод Interactor'а
    \item лог диалога решения
    \item лог Interactor'а
\end{itemize}

\pagebreak

\section{Рекомендации для автоматических проверяющих систем}
\subsection{Стандартные Interactor'ы}
Предоставить множество стандартных Interactor'ов под популярные форматы взаимодействий,
например
\begin{verbatim}
    n
    запрос1
    запрос2
\end{verbatim}

\subsection{Информация о запуске решения}
В любом случае кроме падения Interactor'а следует предоставлять информацию
о неуспешном запуске решения (например, ABNORMAL\_EXIT, TIME\_LIMIT\_EXCEEDED ...).

\pagebreak

\section{Рекомендации разработчикам задач}

\subsection{Типы интерактивных задач}
\begin{enumerate}
    \item Задачи на онлайн запросы

        Это задачи, в которых требуется посчитать ответ для некоторого количества однотипных запросов,
        где следующие запрос дается после ответа на предыдущий.

    \item Интерактивные задачи без вывода ответа

        Здесь нет понятия ответа как такового. Взаимодействие программ прекращается при достижении определенного состояния. Например, сюда относятся все игровые задачи.

    \item Интерактивные задачи с выводом ответа

        В этих задачах программа участника перед завершением должна выводить ответ,
        который в дальнейшем будет проверен программой жюри.
\end{enumerate}

\subsection{Рекомендации по оформлению условий}
TODO: переписать с учётом изменений множества статусов

\begin{itemize}
    \item \textit{Задачи на онлайн запросы} -- так же, как и обычные задачи, но с добавлением информации об интерактивности запросов.
    \item \textit{Другие интерактивные задачи} -- условие должно включать в себя
        \begin{itemize}
            \item Описание задачи
            \item Интерактивный протокол
            \item Формат входных данные
            \item Формат выходных данных
        \end{itemize}

        \begin{itemize}
            \item Interactor выводит данные в формате, описанном в формате входных данных
            \item Несоблюдение формата выходных данных повлечет ошибку PE
            \item Несоблюдение формата интерактивного протокола повлечет ошибку IE % TODO проработать статусы
            \item Неправильный результат программы повлечет ошибку WA
        \end{itemize}

        Стоит отметить, что в задачах 2-го типа нет ответа, но, например, в игровых задачах вывод некорректного хода - это INCORRECT\_REQUEST,
        а проигрыш - WRONG\_ANSWER (может стоит переименовать вердикт для этих задач?).

        Также если участник может тестировать свое решение на сервере,
        то условие должно содержать описание формата теста, которое считывает Interactor.
\end{itemize}

\end{document}
