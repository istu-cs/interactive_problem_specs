\input head

\begin{document}

\section{Типы интерактивных задач}
\begin{enumerate}
    \item Задачи на онлайн запросы.

    Это задачи, в которых требуется посчитать ответ для некоторого количества однотипных запросов, где следующие запрос дается после ответа на предыдущий.

    \item Интерактивные задачи без вывода ответа.

    Здесь нет понятия ответа как такового. Взаимодействие программ прекращается при достижении определенного состояния. Например, сюда относятся все игровые задачи.

    \item Интерактивные задачи с выводом ответа.

    В этих задачах программа участника перед завершением должна выводить ответ, который в дальнейшем будет проверен программой жюри.
\end{enumerate}

\section{Содержание архива}
\begin{enumerate}
    \item
        \begin{enumerate}
            \item generator
            \item checker
            \item validator
            \item jury solution
        \end{enumerate}

    Также в системе должна присутствовать стандартная программа, позволяющая выводить запросы интерактивно. Понятно, тесты для таких задач будут иметь специфических формат.

    \item
        \begin{enumerate}
            \item generator
            \item validator
            \item jury solution
            \item interactor
        \end{enumerate}

   В таких задачах нет ответа, поэтому и checker не нужен.

    \item
        \begin{enumerate}
            \item generator
            \item checker
            \item validator
            \item jury solution
            \item interactor
        \end{enumerate}
\end{enumerate}

\section{Проверка решений}
\begin{enumerate}
    \item система запускает интерактор и решение и перенаправляет их стандартные потоки друг на друга
    \item интерактор считывает тест (input.txt)
    \item общение интерактора и решения
    \item интерактор выводит ответ (output.txt)
    \item чекер проверяет ответ
\end{enumerate}

\section{Возможные ошибки}
\begin{enumerate}
    \item Решение участника работает слишком долго (time-limit exceeded);
    \item Решение участника расходует слишком много памяти (memory-limit exceeded);
    \item Решение участника завершилось с ненулевым кодом возврата (run-time error);
    \item Решение участника вывело запрос в неправильном формате (presentation error);

     Пример: интерактор ждет два числа, а ему пришла строка.

    \item Решение участника нашло неправильный ответ, т. е. от чекера пришел отрицательный вердикт (wrong answer);
    \item Решение участника выводит слишком много запросов (queries-limit exceeded);

    \item Интерактор не может посчитать ответ на запрос программы участника (interaction error).

     Другими словами, у интерактора случился \textit{run-time error}.

     Например, если участник в своей программе обратится к стандартной функции \textit{sqrt()} с отрицательным значением - это RE.

     Аналогично. Если участник попросил интерактор посчитать корень от отрицательного числа - это IE, а не WA, потому что это действие ни как не связано с ответом.
\end{enumerate}

\section{Оформление условия}
\begin{description}
    \item[1] Так же, как и обычные задачи, но с добавлением информации об интерактивности запросов.
    \item[2-3] Описание задачи, интерактивный протокол, формат входных данные, формат выходных данных.
        \begin{itemize}
            \item Интерактор выводит данный в формате, описанном в формате входных данных.
            \item Не соблюдение формата выходных данных повлечет ошибку PE.
            \item Не соблюдение формата интерактивного протокола повлечет ошибку IE.
            \item Неправильный результат программы повлечет ошибку WA.
        \end{itemize}

       Стоит отметить, что в задачах 2-го типа нет ответа, но, например, в игровых задачах вывод некорректного хода - это IE, а проигрыш - WA (может стоит переименовать вердикт для этих задач?).

       Также если участник может тестировать свое решение на сервере, то условие должно содержать описание формата теста, которое считывает интерактор.
\end{description}

\section{Определение вердикта}
Интерактор ждет от решения полноценный запрос (определенное количество строк, байт и т. п.). Если интерактор не смог считать весь запрос - это вердикт TL/ML/RE. Если он считал запрос, то он должен его проверить, и в случае необходимости выдать вердикт WA/QL/IE.

\section{Тестирование задач}
Участнику должна быть предоставлена возможность тестировать свои решение на сервере.
Для этого к задаче должны прилагаться несколько тестов жюри (около 10).
Эти тесты необязательно должны соблюдать формат официальных тестов
(например, в задаче $n = 10^5$, а участнику нужно проверить решение при маленьких n).

После запуска на сервере ему должна быть предоставлена следующая информация:
\begin{itemize}
    \item вердикт
    \item время
    \item память
    \item вывод решения
    \item вывод интерактора
    \item лог диалога решения
    \item лог интерактора
\end{itemize}

\end{document}
